

% Access the author name in the document
% Later in the document \ldots we access the variables again:
% \MyTitle, \MyAuthor, and \MyDate.	
\usepackage{authoraftertitle}
\title{MAT2040: Linear Algebra}
\author{ZHOU, Yuming 121050081}
\newcommand{\COURSE}{MAT2040}
\newcommand{\YEAR}{Summer 2022}



\usepackage{thmbox}

\usepackage{fontspec}
\defaultfontfeatures{Mapping=tex-text,Scale=MatchLowercase}
%\setmainfont{OpenDyslexic3}
\setmainfont{Atkinson Hyperlegible}
\setmonofont{Meslo LG L for Powerline}

\newtheorem[L]{thm}{Theorem}[section]

\newtheorem{cor}[thm]{Corollary}



% update @ 2021
%
% this function resolves error of repeated commands
\newcommand*{\renameenviron}[1]{\expandafter\let\csname renamed-#1\expandafter\endcsname\csname #1\endcsname\expandafter\let\csname endrenamed-#1\expandafter\endcsname\csname end#1\endcsname\expandafter\let\csname #1\endcsname\relax\expandafter\let\csname end#1\endcsname\relax}

\usepackage{amsmath}

\usepackage{cases}
\usepackage{amsfonts}
\usepackage{amssymb}
\usepackage{dsfont} % \mathds
\usepackage{geometry} % page margin
\usepackage{indentfirst}
\usepackage{xcolor}
\definecolor{eggray}{rgb}{0.4, 0.4, 0.35}
\usepackage[colorlinks]{hyperref}
\hypersetup{
colorlinks,
linkcolor={red!50!black},
citecolor={blue!50!black},
urlcolor={blue!80!black}
}

% NOTE: Always put hyperef before gls
\usepackage{glossaries}

\makeglossaries
\glstoctrue
\loadglsentries{gls}
\newcommand{\g}[1]{\gls{#1}}
\newcommand{\gp}[1]{\glspl{#1}}

\usepackage{cleveref}

\usepackage{graphicx} % images
\setkeys{Gin}{width=0.6\linewidth}
\usepackage{wrapfig} % figure wrapped by texts
\usepackage{soul} % \ul
\usepackage{mathrsfs} % mathsrc
\usepackage{caption} % caption*
\usepackage[inline]{enumitem} % change label of enumerate
\usepackage{physics} % \dv and \pdv
\usepackage{gensymb} % \degree
\usepackage{linegoal} % \linegoal returns left width in a line
\usepackage{booktabs} % \toprule in tabular
\usepackage[d]{esvect} % \vv for vec

\usepackage{tikz} % tikz
%\usepackage{tikzit} % with tikzit app

% \usepackage{courier} % font
% \usepackage{listings} % font for code
% \lstset{basicstyle=\footnotesize\ttfamily,breaklines=true}
% \lstset{framextopmargin=50pt,frame=bottomline}
% \usepackage[cache=false]{minted} % minted setup to include code
%\renewcommand{\MintedPygmentize}{/home/jpl/anaconda3/bin/pygmentize}
% \renewcommand{\MintedPygmentize}{/Users/jacky/anaconda3/bin/pygmentize}
% \setminted[python]{linenos, tabsize=2, breaklines}

% for pseudo code
\usepackage{algorithm}
\usepackage{algpseudocode}

\usepackage{chngcntr} % for counterwithin
\usepackage{calc} % get width of contents

%
% The following commands set up the lecnum (lecture number)
% counter and make various numbering schemes work relative
% to the lecture number.
%
\newcounter{lecnum}
\counterwithin{section}{lecnum}
\counterwithin{equation}{lecnum}
\counterwithin{figure}{lecnum}
\counterwithin{table}{lecnum}
\counterwithin{algorithm}{lecnum}
\counterwithin{footnote}{lecnum}

\pagestyle{myheadings}

\newlength{\leclen}
\newcommand{\lecture}[4]{%
\clearpage
\thispagestyle{plain}
\stepcounter{lecnum} % resets every associated counter
\setcounter{lecnum}{#1}%
\begin{center}
\framebox{%
\phantomsection % toc locate to section
\addcontentsline{toc}{part}{\thelecnum\hspace{2em}\space#2}
\parbox[t]{\linewidth-0.5mm}{%
\vspace{2mm}
{\bfseries\COURSE \hfill \YEAR \par}
\vspace{2mm}
{\Large %Lecture #1: #2
\settowidth{\leclen}{Lecture999:}
\begin{minipage}[t]{\leclen}
Lecture #1:
\end{minipage} \begin{minipage}[t]{\linewidth-2mm-\leclen}
#2
\end{minipage}
\par}
\vspace{4mm}
{\itshape Lecturer: #3\hfill Date: #4\par}
\vspace{2mm}
}%
}
\end{center}
\markboth{Lecture #1}{Lecture #1}
}

\newlength{\hwklen}
\newcommand{\hwk}[4]{%
\clearpage
\thispagestyle{plain}
\stepcounter{lecnum} % resets every associated counter
\setcounter{lecnum}{#1}%
\begin{center}
\framebox{%
\phantomsection % toc locate to section
\addcontentsline{toc}{part}{\thelecnum\hspace{2em}\space#2}
\parbox[t]{\linewidth-0.5mm}{%
\vspace{2mm}
{\bfseries\COURSE \hfill \YEAR \par}
\vspace{2mm}
{\Large %Lecture #1: #2
\settowidth{\hwklen}{Assignment999:}
\begin{minipage}[t]{\hwklen}
Assignment #1:
\end{minipage} \begin{minipage}[t]{\linewidth-2mm-\hwklen}
#2
\end{minipage}
\par}
\vspace{4mm}
{\itshape Student: #3\hfill Date: #4\par}
\vspace{2mm}
}%
}
\end{center}
\markboth{Assignment #1}{Assignment #1}
}

% theorem style
\renameenviron{leftbar} % relax \leftbar to make thmbox work
\usepackage{thmbox} % footnote disappears in such envs
\newtheorem[style=M]{definition}{Definition}[lecnum]
\newtheorem[L]{theorem}{Theorem}[lecnum]
\newtheorem[S]{proposition}[theorem]{Proposition}
\newtheorem[S]{lemma}[theorem]{Lemma}
\newtheorem[S]{corollary}[theorem]{Corollary}

%
% Enumerate and Itemize

%
% Enumerate and Itemize

\newlength{\mywidth}

\newcounter{Egnum}[lecnum]
\newenvironment{Eg}{\refstepcounter{Egnum}\par\medskip \noindent \textit{e.g.\theEgnum.} \color{eggray} \rmfamily } {\medskip}
\newenvironment{Eg*}{\par\medskip \noindent \textit{e.g.} \color{eggray} \rmfamily } {\medskip}
\newenvironment{QaA}{\par\medskip \noindent \textit{Q\&A.} \color{eggray} \rmfamily } {\medskip}
\newcommand{\eg}[1]{{\color{eggray}(}{\textit{e.g.} \color{eggray} #1)}}
\newcommand{\ie}[1]{{\color{eggray}(}{\textit{i.e.} \color{eggray} #1)}}
\newcommand{\br}[1]{(#1)}

% DEFINE ADDITIONAL MACROS
\newcommand\note{\textit{Note:\hspace{0.75cm}}}
\newcommand{\TODO}{\colorbox{red}{NOT DOING YET!}}
\newcommand{\question}{\textit{\frame{Question:}\hspace{0.75cm}}}
\newcommand{\claim}{\textit{\frame{Claim:}\hspace{0.75cm}}}
\newcommand{\recall}{\textit{\frame{Recall:}\hspace{0.75cm}}}

\newcommand{\mycell}[2][c]{\begin{tabular}[#1]{@{}c@{}}#2\end{tabular}} % used in tabular to force new line

% DEFINE MATH MACROS
\newcommand{\qed}{\hfill\rule{2mm}{2mm}}
\DeclareMathOperator*{\argmax}{arg\,max}
\DeclareMathOperator*{\argmin}{arg\,min}
\DeclareMathOperator*{\LHS}{LHS}
\DeclareMathOperator*{\RHS}{RHS}
\newcommand{\floor}[1]{\lfloor #1 \rfloor}
\newcommand{\ceil}[1]{\lceil #1 \rceil}
\newcommand{\myrightarrow}[1]{\mathrel{\raisebox{-2pt}{$\xrightarrow{#1}$}}}


% Other commands
\renewcommand\st{\quad\text{s.t.}\quad}
\newcommand\E{\mathbb{E}} % expectation
\newcommand\F{\mathbb{F}} % fields
\newcommand\R{\mathbb{R}} % real
\newcommand\Q{\mathbb{Q}} % quadratic
\newcommand\Z{\mathbb{Z}} % integer
\newcommand\N{\mathbb{N}} % natural
\newcommand\bbC{\mathbb{C}} % complex
\newcommand\bbI{\mathbb{I}} % index function
\newcommand\bbS{\mathbb{S}}
\newcommand\calR{\mathcal{R}} % integrability
\newcommand\calF{\mathcal{F}} % family
\newcommand\calA{\mathcal{A}} % algebra
\newcommand\calB{\mathcal{B}} % Borel algebra
\newcommand\calM{\mathcal{M}}
\newcommand\bfi{\mathbf{i}}
\newcommand\bfj{\mathbf{j}}
\newcommand\bfk{\mathbf{k}}
\newcommand{\diam}{\operatorname{diam}}
\newcommand{\sumin}{\sum_{i=1}^n}
\newcommand{\sumiN}{\sum_{i=1}^N}
\newcommand{\sumiM}{\sum_{i=1}^M}
\newcommand{\sumjn}{\sum_{j=1}^n}
\newcommand{\sumjm}{\sum_{j=1}^m}
\newcommand{\sumil}{\sum_{i=1}^\ell}
\newcommand{\intab}{\int_{a}^{b}}

% Algebra
\newcommand{\GL}{\operatorname{GL}}
\newcommand{\SL}{\operatorname{SL}}
\newcommand{\Sym}{\operatorname{Sym}}
\newcommand{\lcm}{\operatorname{lcm}}
\newcommand{\sgn}{\operatorname{sgn}}
\newcommand{\im}{\operatorname{im}}
\renewcommand{\ker}{\operatorname{ker}}
\newcommand{\opNull}{\operatorname{null}}
\newcommand{\range}{\operatorname{range}}

% Analysis
\newcommand\simpleS{\boldsymbol{s}} % simple function
\newcommand{\ptwto}{\mathrel{\raisebox{-4pt}{$\xrightarrow{\text{pointwise}}$}}}
\newcommand{\unfto}{\mathrel{\raisebox{-4pt}{$\xrightarrow{\text{uniformly}}$}}}

% Machine learning
\newcommand\LossL{\mathcal{L}} % loss
\newcommand\RiskR{\mathcal{R}}
\newcommand\calN{\mathcal{N}} % normal distribution
\newcommand\calG{\mathcal{G}}
\newcommand\Var{\operatorname{var}} % variance
\newcommand\Cov{\operatorname{cov}} % covariance
\newcommand\Bias{\operatorname{Bias}} % bias
\newcommand\MSE{\operatorname{MSE}} % MSE loss
\newcommand{\MultiGaussianPDF}[1]{\frac{1}{\sqrt{(2\pi)^d|\Sigma_{#1}|}} \exp\left(-\frac12 (x-\mu_{#1})^T \Sigma_{#1}^{-1} (x-\mu_{#1})\right)}
%\newcommand{\algorithmSplit}[1]{\end{algorithmic} #1 \begin{algorithmic}[1]} % OLD TO BE DELETED
\newcommand{\idx}[1]{^{(#1)}} % superscript

% Linear Algebra
\newcommand{\Row}[1]{R_{#1}} % row
\newcommand{\x}[1]{x_{#1}} % x unknown
\newenvironment{amatrix}[1]{%
  \left[\begin{array}{@{}*{#1}{c}|c@{}}
}{%
  \end{array}\right]
}
\newcommand{\ans}{\text{Answer:}}
\newcommand{\rop}[1]{\stackrel{\substack{#1}}{\to}}